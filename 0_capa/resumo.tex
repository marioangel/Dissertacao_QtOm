% -- Resumo -------------------------------------------------------------------
\chapter*{Resumo}
\thispagestyle{empty}

%{\onehalfspacing

O processo de desenvolvimento de software para sistemas embarcados tem crescido rapidamente, o que na maioria das vezes acarreta num aumento da complexidade associada a esse tipo de projeto. Como consequ�ncia, as empresas eletr�nicas de consumo costumam investir diversos recursos em mecanismos de verifica��o r�pida e autom�tica, com o intuito de criar sistemas robustos e reduzir as taxas de produtos reparados. Al�m disso, a redu��o no tempo de desenvolvimento e na robustez dos sistemas criados podem ser alcan�ados atrav�s de frameworks multi-plataformas, tais como Qt, que oferece um conjunto de softwares confi�veis para v�rios dispositivos. Desta forma, o trabalho atual prop�e uma vers�o simplificada do framework Qt, cujo integrado a um verificador baseado em Satisfiability Modulo Theories(SMT), denominado Efficient SMT-Based Bounded Model Checker(ESBMC++), verifica aplica��es reais que ultilizam Qt, apresentando uma taxa de sucesso de 89\%, para o pacote de benchmark desenvolvido. Com a vers�o simplificada do framework Qt proposta, tamb�m foi feito uma avalia��o ultilizando outros verificadores que se encontram no estado da arte para programas em C++. Dessa maneira, evidenciando-se que a metodologia proposta � a primeira a verificar formalmente aplica��es baseadas no framework Qt al�m de possuir um potencial para desenvolver novos rumos para a verifica��o de software de c�digo port�til.


\vspace*{\stretch{1}} %texto se adapta para ocupar toda a p�gina

\noindent \textsf{Palavras-chave:} engenharia de software, Qt Framework, bounded model checking, verifica��o formal.
%}

\cleardoublepage
