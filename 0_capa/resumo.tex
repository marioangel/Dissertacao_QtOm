% -- Resumo -------------------------------------------------------------------
\chapter*{Resumo}
\thispagestyle{empty}

%{\onehalfspacing

O processo de desenvolvimento de software para sistemas embarcados tem crescido rapidamente, o que na maioria das vezes acarreta num aumento da complexidade associada a esse tipo de projeto. Como consequ�ncia, as empresas de eletr�nica de consumo costumam investir recursos em mecanismos de verifica��o r�pida e autom�tica, com o intuito de criar sistemas robustos e assim reduzir as taxas de \textit{recall} de produtos. Al�m disso, a redu��o no tempo de desenvolvimento e na robustez dos sistemas criados podem ser alcan�ados atrav�s de \textit{frameworks} multi-plataformas, tais como Qt, que oferece um conjunto de bibliotecas (gr�ficas) confi�veis para v�rios dispositivos. Desta forma, o trabalho atual prop�e uma vers�o simplificada do \textit{framework} Qt, integrado a um verificador baseado nas teorias do m�dulo da satisfatibilidade, denominado \textit{Efficient SMT-Based Bounded Model Checker} (ESBMC++), o qual verifica aplica��es reais que ultilizam o Qt, apresentando uma taxa de sucesso de $89$\%, para os \textit{benchmarks} desenvolvidos. Com a vers�o simplificada do \textit{framework} Qt proposta, tamb�m foi feito uma avalia��o ultilizando outros verificadores que se encontram no estado da arte para programas em C++. Dessa maneira, evidenciando-se que a metodologia proposta � a primeira a verificar formalmente aplica��es baseadas no \textit{framework} Qt, al�m de possuir um potencial para desenvolver novas frentes para a verifica��o de c�digo port�til.


\vspace*{\stretch{1}} %texto se adapta para ocupar toda a p�gina

\noindent \textsf{Palavras-chave:} Engenharia de Software, \textit{Framework} Qt, Verifica��o Formal, \textit{Bounded Model Checking}.
%}

\cleardoublepage
